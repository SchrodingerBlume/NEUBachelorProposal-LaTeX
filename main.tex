% !TeX encoding = UTF-8
% !TeX program = xelatex
% 东北大学本科毕业设计(论文)开题报告 LaTeX 模板
% 使用说明:
% 1. 确保 NEUBachelorProposal.cls 文件与本文件在同一目录
% 2. 使用 XeLaTeX 编译
% 3. 使用 Biber 作为文献工具
% 4. 按照 XeLaTeX -> Biber -> XeLaTeX -> XeLaTeX 的顺序编译
\documentclass{NEUBachelorProposal}

% ==================== 封面信息设置 ====================
% 请根据实际情况修改以下信息
\proposaltitle{在这里输入标题,模板版本\Version}% 论文题目

% 如果标题只有一行,但是却出现了多余的空白或第二行线
% 则请把 \forcetitlenofilloff 最后的 off 改成 on
\forcetitlenofilloff

% 如果标题仅一行,则是居中对齐
% 如果标题有两行,则改为左对齐
% 把 \forcecentertitleoff 最后的 off 改成 on
% 则在有两行标题时,也会居中对齐
\forcecentertitleoff

\school{生命科学与健康学院}% 学院名称
\major{生物工程}% 专业名称
\studentid{20227523}% 学号
\grade{2022级}% 年级
\studentname{薛定谔}% 学生姓名
\supervisor{毛毛}% 指导教师
\submityear{2025}% 年份
\submitmonth{3}% 月份
\submitday{3}% 日期

\begin{document}

% ==================== 生成封面 ====================
\makecover

% ==================== 正文内容 ====================
\linespread{1.625}% 正文预设行距 1.625,对应 Word 1.5 倍行距
%%%%%%%%%%%%%%%%%%%%%%%%%%%%%%%%%%%%%%%%%%%%%%%%%%%%
% 选题意义
\mokuai{选\\题\\意\\义}{%
{\tips{(从课题的来源,目的及意义来说明选题的合理性,从技术方面来论述选题的理论研究价值,从经济、社会效益等方面论述选题的实际应用价值。)}}

\suojin \textbf{首先声明:本模板不包含\LaTeX\ 的基本使用教程。 }

\suojin 编译器请选择 XeLaTeX,文献工具请选择 biber。

\suojin 每个模块使用\texttt{\textbackslash mokuai}[<字号>]\{<左侧文字>\}\{<正文内容>\}得到,其中,字号参数缺省时,默认使用小四号字体。例如想要使用五号字,就使用\texttt{\textbackslash mokuai[5]\{\}\{\}};想要使用小三号字,就使用\texttt{\textbackslash mokuai[-3]\{\}\{\}}。

\suojin 左侧文字使用\texttt{\textbackslash\textbackslash}来换行。
}
%%%%%%%%%%%%%%%%%%%%%%%%%%%%%%%%%%%%%%%%%%%%%%%%%%%%
% 国内外研究现状概述
\mokuai{国\\内\\外\\研\\究\\现\\状\\概\\述}{%
\tips{(根据调研结果从理论、技术、具体应用、发展方向等方面阐述当前课题的国内外研究或实现现状,进而说明当前存在哪些不足……)}

\suojin 像这样在段首使用\texttt{\textbackslash suojin}命令实现首行缩进。

像这样不在段首使用\texttt{\textbackslash suojin}命令则不会首行缩进。

\suojin 每次新起一段文字时,应当在代码中空一行。

\section{各级标题命令可以使用}

\subsection{二级}

\subsubsection{三级}

\paragraph{四级标题默认设置是不换行的}

这是正文,与\texttt{\textbackslash paragraph}之间没有换行。
}
%%%%%%%%%%%%%%%%%%%%%%%%%%%%%%%%%%%%%%%%%%%%%%%%%%%%
% 主要研究内容
\mokuai{主\\要\\研\\究\\内\\容}{%
\tips{(从毕业设计具体从事的工作进行描述,说明自己毕业设计所做的内容。)}

\suojin\texttt{\textbackslash mokuai}命令的实现基于表格环境,因此许多常规的\LaTeX\ 命令无法使用。例如\texttt{\textbackslash verb}命令就不可用,但这无伤大雅,必须指出,在开题报告中写代码实在不是明智之选。

\suojin 首先介绍图的使用,基本框架是:\texttt{\textbackslash tu\{<图片宽度>\}\{<图片路径>\}},其中图片宽度不得超过\texttt{13.825cm};图的题注命令格式是:\texttt{\textbackslash tizhu\{tu\}\{<题注内容>\}\textbackslash label\{<图标签>\}}。看图\ref{fig:flower}这个例子:

\vskip 2em % 2 字高刚好齐底,即再大了,这个模块就会跑到下一页
}

% 跨页后若要续上个模块,将左侧文字留空
\mokuai{}% 留空
{
\tu{4cm}{figures/flower.png}
\tizhu{tu}{蓝天下的一朵花}\label{fig:flower}

\suojin 跨页后若要续上个模块,应新起一个模块,并将左侧文字留空(当然,你要是想重复左侧文字的话也可以)。可以在每页插入若干\texttt{\textbackslash vskip}命令,以调整底部横线至足够美观的位置。例如上一页最后就插入了一个\texttt{\textbackslash vkip 2em}。

\suojin 接下来介绍表格的使用。由于\texttt{\textbackslash mokuai}中不支持继续嵌套表格,只能通过\texttt{tabular(x)}环境结合\texttt{minipage}环境来实现,本模板已将其封装为一个宏命令,其框架为:

\texttt{\textbackslash tizhu\{biao\}\{<题注内容>\}\textbackslash label\{<表标签>\}}

\texttt{\textbackslash biao\{<列数>\}\{}

\texttt{表头1 \& 表头2 \textbackslash\textbackslash~\textbackslash midrule \% <-- 表头要加}

\texttt{内容1 \& 内容2 \textbackslash\textbackslash}

\texttt{\}}

\suojin 如表\ref{tab:motto}所示,单元格内可以套上一层\texttt{\textbackslash makecell[c]\{\}},并结合\texttt{\textbackslash\textbackslash}来换行。

\tizhu{biao}{东大校训与生科院训}\label{tab:motto}
\biao{2}{
东大校训 & 生科院训 \\ \midrule
\makecell[c]{自强不息\\知行合一} & 笃志近思,厚德敦行 \\
}
}
%%%%%%%%%%%%%%%%%%%%%%%%%%%%%%%%%%%%%%%%%%%%%%%%%%%%
% 拟采用的研究思路
\mokuai{拟采用\\的研究\\思\hspace{1em}路}{%
\tips{(方法、技术路线、可行性论证等。)}

\suojin “工作进度安排”有自己专属的命令,其格式为:

\texttt{\textbackslash jindu\{}

\texttt{ <阶段1> \& <起止日期1> \& <任务1> \& <阶段成果1> \textbackslash\textbackslash~\textbackslash hline}

\texttt{ <阶段2> \& <起止日期2> \& <任务2> \& <阶段成果2> \textbackslash\textbackslash~\textbackslash hline}

\texttt{……\} \%~最后一行可以不加\textbackslash hline}
}

% 续上一模块
\mokuai{}{%
\suojin 进度模块中如需换行,应使用\texttt{\textbackslash newline}命令,具体示例见对应模块。
}

%%%%%%%%%%%%%%%%%%%%%%%%%%%%%%%%%%%%%%%%%%%%%%%%%%%%
% 工作进度安排
\jindu{%
  1  & 3月2日—3月15日\newline(1-2周)   &  与教师沟通毕设任务 & 开题报告 \\ \hline
  2  & 3月16日—3月22日\newline(3-3周)  &  查阅文献资料 &  论文综述 \\ \hline
  3  & 3月23日—4月5日\newline(4-5周)   &  构建缺陷型菌株 & 检测数据结果图 \\ \hline
  4  & 3月23日—4月5日\newline(4-5周)   &  构建过表达菌株 & 检测数据结果图、提交中期报告 \\ \hline
  5  & 4月27日—5月17日\newline(9-11周) &  整理论文资料,\newline 撰写论文 & 论文全文 \\ \hline
  6  & 5月18日—5月24日\newline(12-12周)&  整理论文资料 & 毕业设计手册 \\ \hline
  7  & 5月24日—5月31日\newline(13-13周)&  修改、审核论文 & 排版格式正确的论文初稿全文 \\ \hline
  8  & 6月1日—6月7日\newline (14-14周)&  准备答辩 & 各种答辩材料 \\ % 不加\hline,加也行
}

% 参考文献目录
\mokuai{参考\\文献\\目录}{%
\suojin 参考文献\texttt{bib}信息统一放在根目录下的\texttt{references.bib}中。在文中使用\texttt{\textbackslash cite\{<bibkey>\}}来引用\cite{1_book_example1};同时引用多个文献时,把所有\texttt{<bibkey>}放到一个命令中,用半角逗号隔开\cite{2_book_example2,3_conference_example1,4_collection_example1,5_thesis_example1};正文中如需对引文进行阐述时,应使用\texttt{\textbackslash parencite\{\}}命令代替\texttt{\textbackslash cite\{\}}命令,且用法不变,如:\parencite{1_book_example1,2_book_example2,7_report_example1,8_report_example2}。

\suojin 打印文献使用\texttt{\textbackslash printbibrange\{<起始编号>\}\{<结束编号>\}},只有引用过才能被打印\cite{9_patent_example1,11_inbook_example1,12_inconference_example1,13_periodical_example1,14_periodical_example3,15_journal_example1,16_news_example1,17_online_book_example1,18_online_journal_example1,19_online_news_example1,20_dbmt}。例如\texttt{\textbackslash printbibrange\{1\}\{1\}}就会打印第1篇文献。

\printbibrange{1}{1}
}
% 续上一模块
\mokuai{}{%

\tips{再例如\texttt{\textbackslash printbibrange\{2\}\{11\}}就会打印第2至11篇文献。}

\printbibrange{2}{11}

\vskip 0.75em% 插入空白至齐底

\tips{应当合理调整打印范围,并结合\texttt{\textbackslash vskip}以平衡分页的合理性与美观性。}
}
% 续上一模块
\mokuai{}{%
\printbibrange{12}{18}

\suojin 会议纪要太过复杂,使用\LaTeX\ 来实现反而增加不便。请在根目录下找到\texttt{huiyijiyao.doc},在 Word 中填写后,输出为同名 pdf 文件,然后以之覆盖根目录中已有的\texttt{huiyijiyao.pdf}。\texttt{\textbackslash huiyijiyao}命令用于插入这个 pdf。

\vskip 14em% 插入空白至齐底

\tips{应当合理调整打印范围,并结合\texttt{\textbackslash vskip}以平衡分页的合理性与美观性。}
}
%%%%%%%%%%%%%%%%%%%%%%%%%%%%%%%%%%%%%%%%%%%%%%%%%%%%
% 会议纪要
\huiyijiyao

\end{document}
