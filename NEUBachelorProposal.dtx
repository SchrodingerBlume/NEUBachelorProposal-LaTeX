% \iffalse meta-comment
%
% NEUBachelorProposal.dtx
% Copyright 2025 SchrodingerBlume
%
% This work may be distributed and/or modified under the
% conditions of the LaTeX Project Public License, either version 1.3
% of this license or (at your option) any later version.
% The latest version of this license is in
%   http://www.latex-project.org/lppl.txt
% and version 1.3 or later is part of all distributions of LaTeX
% version 2005/12/01 or later.
%
% This work has the LPPL maintenance status `maintained'.
%
% The Current Maintainer of this work is SchrodingerBlume.
%
% This work consists of the files NEUBachelorProposal.dtx and NEUBachelorProposal.ins
% and the derived file NEUBachelorProposal.cls.
%
% \fi
%
% \iffalse
%<*driver>
\ProvidesFile{NEUBachelorProposal.dtx}
%</driver>
%<class>\NeedsTeXFormat{LaTeX2e}
%<class>\ProvidesClass{NEUBachelorProposal}[2025/11/19 V-1.0.0 A Bachelor Proposal LaTeX Template for NEU students]
%
%<*driver>
\documentclass{ctexart}
\usepackage{hypdoc}
\usepackage{geometry}
\geometry{left=3cm,right=3cm,top=2.5cm,bottom=2.5cm}
\usepackage{xcolor}
\usepackage{listings}
\lstset{
  basicstyle=\ttfamily\small,
  breaklines=true,
  frame=single,
  backgroundcolor=\color{gray!10}
}
\EnableCrossrefs
\CodelineIndex
\RecordChanges
\MakeShortVerb{\|}
\begin{document}
  \DocInput{NEUBachelorProposal.dtx}
  \PrintChanges
  \PrintIndex
\end{document}
%</driver>
% \fi
%
% \CheckSum{0}
%
% \CharacterTable
%  {Upper-case    \A\B\C\D\E\F\G\H\I\J\K\L\M\N\O\P\Q\R\S\T\U\V\W\X\Y\Z
%   Lower-case    \a\b\c\d\e\f\g\h\i\j\k\l\m\n\o\p\q\r\s\t\u\v\w\x\y\z
%   Digits        \0\1\2\3\4\5\6\7\8\9
%   Exclamation   \!     Double quote  \"     Hash (number) \#
%   Dollar        \$     Percent       \%     Ampersand     \&
%   Acute accent  \'     Left paren    \(     Right paren   \)
%   Asterisk      \*     Plus          \+     Comma         \,
%   Minus         \-     Point         \.     Solidus       \/
%   Colon         \:     Semicolon     \;     Less than     \<
%   Equals        \=     Greater than  \>     Question mark \?
%   Commercial at \@     Left bracket  \[     Backslash     \\
%   Right bracket \]     Circumflex    \^     Underscore    \_
%   Grave accent  \`     Left brace    \{     Vertical bar  \|
%   Right brace   \}     Tilde         \~}
%
% \changes{v1.0.0}{2025/11/19}{Initial version}
%
% \GetFileInfo{NEUBachelorProposal.dtx}
%
% \DoNotIndex{\newcommand,\renewcommand,\RequirePackage,\LoadClass}
% \DoNotIndex{\begin,\end,\if,\else,\fi,\iffalse,\ifthen}
%
% \title{东北大学本科毕业设计(论文)开题报告 \LaTeX{} 模板\\
%        \texttt{NEUBachelorProposal} 文档类\thanks{本文档对应
%        \texttt{NEUBachelorProposal}~\fileversion,
%        最后修改于 \filedate。}}
% \author{SchrodingerBlume}
% \date{\filedate}
%
% \maketitle
%
% \begin{abstract}
% \noindent
% 本文档类为东北大学本科毕业设计(论文)开题报告提供了一个符合学校规范的 \LaTeX{} 模板。
% 该模板支持 XeLaTeX 编译器,并集成了中文字体自动检测、参考文献管理等功能。
% \end{abstract}
%
% \tableofcontents
%
% \section{简介}
%
% \texttt{NEUBachelorProposal} 是为东北大学本科生毕业设计开题报告设计的 \LaTeX{} 文档类。
% 该模板严格遵循学校的格式要求,并提供了丰富的自定义命令以简化报告的撰写过程。
%
% \subsection{主要特性}
%
% \begin{itemize}
%   \item 自动检测并配置中文字体(支持 Windows、macOS 及 Linux 系统)
%   \item 符合国标 GB7714-2015 的参考文献格式
%   \item 提供封面、模块、图表、工作进度等专用命令
%   \item 支持数学公式的精确排版
%   \item 完善的页面布局和样式设置
% \end{itemize}
%
% \subsection{编译环境}
%
% \begin{itemize}
%   \item 编译器:XeLaTeX
%   \item 文献工具:Biber
%   \item 编译顺序:XeLaTeX → Biber → XeLaTeX → XeLaTeX
% \end{itemize}
%
% \section{使用说明}
%
% \subsection{基本用法}
%
% 在文档开头使用本文档类:
%
% \begin{lstlisting}
% \documentclass{NEUBachelorProposal}
% \end{lstlisting}
%
% \subsection{封面信息设置}
%
% 使用以下命令设置封面信息:
%
% \begin{lstlisting}
% \proposaltitle{论文题目}
% \school{学院名称}
% \major{专业名称}
% \studentid{学号}
% \grade{年级}
% \studentname{学生姓名}
% \supervisor{指导教师}
% \submityear{年份}
% \submitmonth{月份}
% \submitday{日期}
% \end{lstlisting}
%
% 然后使用 |\makecover| 命令生成封面。
%
% \subsection{内容模块}
%
% 使用 |\mokuai| 命令创建内容模块:
%
% \begin{lstlisting}
% \mokuai{左侧标题}{正文内容}
% \end{lstlisting}
%
% 可选参数用于指定字号,如 |\mokuai[-3]{...}{...}| 使用小三号字。
%
% \subsection{图表插入}
%
% 插入图片:
%
% \begin{lstlisting}
% \tu{宽度}{图片路径}
% \tizhu{tu}{图片说明}\label{fig:label}
% \end{lstlisting}
%
% 插入表格:
%
% \begin{lstlisting}
% \tizhu{biao}{表格说明}\label{tab:label}
% \biao{列数}{
%   表头1 & 表头2 \\ \midrule
%   内容1 & 内容2 \\
% }
% \end{lstlisting}
%
% \subsection{工作进度安排}
%
% 使用 |\jindu| 命令创建工作进度表:
%
% \begin{lstlisting}
% \jindu{
%   1 & 起止日期 & 任务 & 阶段成果 \\ \hline
%   2 & 起止日期 & 任务 & 阶段成果 \\ \hline
% }
% \end{lstlisting}
%
% \subsection{参考文献}
%
% 引用文献使用 |\cite{key}| 或 |\parencite{key}|,打印文献列表使用:
%
% \begin{lstlisting}
% \printbibrange{起始编号}{结束编号}
% \end{lstlisting}
%
% \StopEventually{\PrintChanges\PrintIndex}
%
% \section{实现}
%
% \subsection{文档类声明}
%
%    \begin{macrocode}
%<*class>
% 东北大学本科毕业设计(论文)开题报告LaTeX模板样式类
\NeedsTeXFormat{LaTeX2e}
\ProvidesClass{NEUBachelorProposal}[2025/11/19 V-1.0.0 A Bachelor Proposal LaTeX Template for NEU students]
\providecommand{\Version}{V-1.0.0}
\LoadClass[UTF8,heading=true,12pt,a4paper,fontset=none]{ctexart}
%    \end{macrocode}
%
% \subsection{基本宏包}
%
%    \begin{macrocode}
\RequirePackage{array}
\RequirePackage{multirow}
\RequirePackage{makecell}
\RequirePackage{amsmath,amssymb}
\RequirePackage{enumitem}
\RequirePackage{float}
\raggedbottom
%    \end{macrocode}
%
% \subsection{页面设置}
%
%    \begin{macrocode}
\RequirePackage{geometry}
\geometry{
    left=2.2cm,
    right=2.2cm,
    top=2.54cm,
    bottom=2.54cm,
    headheight=15pt,
    headsep=10pt,
    footskip=20.4pt
}
%    \end{macrocode}
%
% \subsection{全局设置}
%
% 设置表格和段落的默认样式。
%
%    \begin{macrocode}
\renewcommand{\arraystretch}{1.3}
\setlength{\tabcolsep}{6pt}
\setlength{\parskip}{0.5em}
%    \end{macrocode}
%
% \subsection{字体设置}
%
% 定义伪粗体和伪斜体参数。
%
%    \begin{macrocode}
\newcommand{\fakeboldvalue}{2.17}
\newcommand{\fakeslantvalue}{0.333}
%    \end{macrocode}
%
% 定义布尔变量用于字体检测。
%
%    \begin{macrocode}
\newif\iffound@system
\newif\iffound@font
\found@systemfalse
%    \end{macrocode}
%
% \subsubsection{Windows 字体}
%
% 优先检测 Windows 系统字体。
%
%    \begin{macrocode}
\IfFontExistsTF{SimSun}{
  \setCJKmainfont{SimSun}[
    AutoFakeBold  = \fakeboldvalue,
    AutoFakeSlant = \fakeslantvalue,
  ]
  \setCJKfamilyfont{zhsong}{SimSun}[
    AutoFakeBold  = \fakeboldvalue,
    AutoFakeSlant = \fakeslantvalue,
  ]
  \setCJKfamilyfont{zhhei}{SimHei}[
    AutoFakeBold  = \fakeboldvalue,
    AutoFakeSlant = \fakeslantvalue,
  ]
  \setCJKfamilyfont{zhkai}{KaiTi}[
    AutoFakeBold  = \fakeboldvalue,
    AutoFakeSlant = \fakeslantvalue,
  ]
  \setCJKfamilyfont{zhfs}{FangSong}[
    AutoFakeBold  = \fakeboldvalue,
    AutoFakeSlant = \fakeslantvalue,
  ]
  \setCJKmonofont{FangSong}[
    AutoFakeBold  = \fakeboldvalue,
    AutoFakeSlant = \fakeslantvalue,
  ]
  \found@systemtrue
}{
%    \end{macrocode}
%
% \subsubsection{macOS 字体}
%
% 检测 macOS 系统字体。
%
%    \begin{macrocode}
  \IfFileExists{/System/Library/Fonts/Menlo.ttc}{
    \IfFontExistsTF{Songti SC}{
      \setCJKmainfont{Songti SC}[
        AutoFakeBold  = \fakeboldvalue,
        AutoFakeSlant = \fakeslantvalue,
      ]
      \setCJKfamilyfont{zhsong}{Songti SC}[
        AutoFakeBold  = \fakeboldvalue,
        AutoFakeSlant = \fakeslantvalue,
      ]
      \setCJKfamilyfont{zhhei}{Heiti SC}[
        AutoFakeBold  = \fakeboldvalue,
        AutoFakeSlant = \fakeslantvalue,
      ]
      \setCJKfamilyfont{zhkai}{Kaiti SC}[
        AutoFakeBold  = \fakeboldvalue,
        AutoFakeSlant = \fakeslantvalue,
      ]
      \setCJKfamilyfont{zhfs}{STFangsong}[
        AutoFakeBold  = \fakeboldvalue,
        AutoFakeSlant = \fakeslantvalue,
      ]
      \setCJKmonofont{STFangsong}[
        AutoFakeBold  = \fakeboldvalue,
        AutoFakeSlant = \fakeslantvalue,
      ]
      \found@systemtrue
    }{}
  }{}
}
%    \end{macrocode}
%
% \subsubsection{其他系统字体}
%
% 对于其他系统,依次尝试 Noto、思源和 Fandol 字体。
%
%    \begin{macrocode}
\iffound@system\else
  \found@fontfalse
  \IfFontExistsTF{Noto Serif CJK SC}{
    \setCJKmainfont{Noto Serif CJK SC}[
      AutoFakeBold  = \fakeboldvalue,
      AutoFakeSlant = \fakeslantvalue,
    ]
    \setCJKfamilyfont{zhsong}{Noto Serif CJK SC}[
      AutoFakeBold  = \fakeboldvalue,
      AutoFakeSlant = \fakeslantvalue,
    ]
    \setCJKfamilyfont{zhhei}{Noto Sans CJK SC}[
      AutoFakeBold  = \fakeboldvalue,
      AutoFakeSlant = \fakeslantvalue,
    ]
    \setCJKfamilyfont{zhkai}{FandolKai}[
      Extension     = .otf,
      UprightFont   = *-Regular,
      AutoFakeBold  = \fakeboldvalue,
      AutoFakeSlant = \fakeslantvalue,
    ]
    \setCJKfamilyfont{zhfs}{FandolFang}[
      Extension     = .otf,
      UprightFont   = *-Regular,
      AutoFakeBold  = \fakeboldvalue,
      AutoFakeSlant = \fakeslantvalue,
    ]
    \setCJKmonofont{FandolFang}[
      Extension     = .otf,
      UprightFont   = *-Regular,
      AutoFakeBold  = \fakeboldvalue,
      AutoFakeSlant = \fakeslantvalue,
    ]
    \found@fonttrue
  }{}
  \iffound@font\else
    \IfFontExistsTF{Source Han Serif SC}{
      \setCJKmainfont{Source Han Serif SC}[
        AutoFakeBold  = \fakeboldvalue,
        AutoFakeSlant = \fakeslantvalue,
      ]
      \setCJKfamilyfont{zhsong}{Source Han Serif SC}[
        AutoFakeBold  = \fakeboldvalue,
        AutoFakeSlant = \fakeslantvalue,
      ]
      \setCJKfamilyfont{zhhei}{Source Han Sans SC}[
        AutoFakeBold  = \fakeboldvalue,
        AutoFakeSlant = \fakeslantvalue,
      ]
      \setCJKfamilyfont{zhkai}{FandolKai}[
        Extension     = .otf,
        UprightFont   = *-Regular,
        AutoFakeBold  = \fakeboldvalue,
        AutoFakeSlant = \fakeslantvalue,
      ]
      \setCJKfamilyfont{zhfs}{FandolFang}[
        Extension     = .otf,
        UprightFont   = *-Regular,
        AutoFakeBold  = \fakeboldvalue,
        AutoFakeSlant = \fakeslantvalue,
      ]
      \setCJKmonofont{FandolFang}[
        Extension     = .otf,
        UprightFont   = *-Regular,
        AutoFakeBold  = \fakeboldvalue,
        AutoFakeSlant = \fakeslantvalue,
      ]
      \found@fonttrue
    }{}
  \fi
  \iffound@font\else
    \setCJKmainfont{FandolSong}[
      Extension     = .otf,
      UprightFont   = *-Regular,
      AutoFakeBold  = \fakeboldvalue,
      AutoFakeSlant = \fakeslantvalue,
    ]
    \setCJKfamilyfont{zhsong}{FandolSong}[
      Extension     = .otf,
      UprightFont   = *-Regular,
      AutoFakeBold  = \fakeboldvalue,
      AutoFakeSlant = \fakeslantvalue,
    ]
    \setCJKfamilyfont{zhhei}{FandolHei}[
      Extension     = .otf,
      UprightFont   = *-Regular,
      AutoFakeBold  = \fakeboldvalue,
      AutoFakeSlant = \fakeslantvalue,
    ]
    \setCJKfamilyfont{zhkai}{FandolKai}[
      Extension     = .otf,
      UprightFont   = *-Regular,
      AutoFakeBold  = \fakeboldvalue,
      AutoFakeSlant = \fakeslantvalue,
    ]
    \setCJKfamilyfont{zhfs}{FandolFang}[
      Extension     = .otf,
      UprightFont   = *-Regular,
      AutoFakeBold  = \fakeboldvalue,
      AutoFakeSlant = \fakeslantvalue,
    ]
    \setCJKmonofont{FandolFang}[
      Extension     = .otf,
      UprightFont   = *-Regular,
      AutoFakeBold  = \fakeboldvalue,
      AutoFakeSlant = \fakeslantvalue,
    ]
  \fi
\fi
%    \end{macrocode}
%
% 定义中文字体命令。
%
%    \begin{macrocode}
\newcommand*{\songti}{\CJKfamily{zhsong}}
\newcommand*{\heiti}{\CJKfamily{zhhei}}
\newcommand*{\kaishu}{\CJKfamily{zhkai}}
\newcommand*{\fangsong}{\CJKfamily{zhfs}}
\setmainfont{Times New Roman}
%    \end{macrocode}
%
% \subsection{数学字体设置}
%
%    \begin{macrocode}
\RequirePackage{unicode-math}
\xeCJKsetup{CJKmath=true}
\setmathfont{XITS Math}[StylisticSet = 8]
\setmathfont{XITS Math}[StylisticSet = 1, range = {cal,bfcal}]
\DeclareRobustCommand\bm[1]{{\symbfit{#1}}}
\DeclareRobustCommand\boldsymbol[1]{{\symbfit{#1}}}
\unimathsetup{
  math-style    = ISO,
  bold-style    = ISO,
  nabla         = upright,
  partial       = upright,
}
\removenolimits{\int\iint\iiint\iiiint\oint\oiint\oiiint}
\RequirePackage{upgreek}
\providecommand{\dd}{\mathop{}\!\mathrm{d}}
\providecommand{\ee}{\mathrm{e}}
\providecommand{\ii}{\mathrm{i}}
\providecommand{\jj}{\mathrm{j}}
\renewcommand\mathellipsis{\cdots}
\AtBeginDocument{
\renewcommand{\Re}{\operatorname{Re}}
\renewcommand{\Im}{\operatorname{Im}}
}
\IfFontExistsTF{SimSun}{
    \setmathfont{SimSun}[range={"2225}]
}{
    \IfFontExistsTF{FandolSong-Regular}{
    \setmathfont{FandolSong-Regular}[range={"2225}]}
{}}
\setmathfont{Times New Roman}[range = {up/{latin,Latin}}]
\IfFontExistsTF{Times New Roman Bold}{
  \setmathfont{Times New Roman Bold}[range = {bfup/{latin,Latin}}]
}{
  \setmathfont{Times New Roman}[range = {bfup/{latin,Latin}}, FakeBold = 2.17]
}
\IfFontExistsTF{Times New Roman Italic}{
  \setmathfont{Times New Roman Italic}[range = {it/{latin,Latin}}]
}{
  \setmathfont{Times New Roman}[range = {it/{latin,Latin}}, FakeSlant = 0.33]
}
\IfFontExistsTF{Times New Roman Bold Italic}{
  \setmathfont{Times New Roman Bold Italic}[range = {bfit/{latin,Latin}}]
}{
  \IfFontExistsTF{Times New Roman Bold}{
    \setmathfont{Times New Roman Bold}[range = {bfit/{latin,Latin}}, FakeSlant = 0.33]
  }{
    \setmathfont{Times New Roman}[range = {bfit/{latin,Latin}}, FakeBold = 2.17, FakeSlant = 0.33]
  }
}
\setmathfont{Times New Roman}[range={"03B5, "03C2, "03C6, "03B1, "03B2, "03B3, "03B4, "03B5, "03B6, "03B7, "03B8, "03B9, "03BA, "03BB, "03BC, "03BD, "03BE, "03BF, "03C0, "03C1, "03C3, "03C4, "03C5, "03C6, "03C7, "03C8, "03C9}]
\DeclareMathAlphabet{\mathcal}{OMS}{cmsy}{m}{n}
\let\mathbb\relax
\DeclareMathAlphabet{\mathbb}{U}{msb}{m}{n}
\RequirePackage{newunicodechar}
\newunicodechar{℃}{\unit{\celsius}}
\newunicodechar{Ⅰ}{I}
\newunicodechar{Ⅱ}{II}
\newunicodechar{Ⅲ}{III}
\newunicodechar{Ⅳ}{IV}
\newunicodechar{Ⅴ}{V}
\newunicodechar{Ⅵ}{VI}
\newunicodechar{Ⅶ}{VII}
\newunicodechar{Ⅷ}{VIII}
\newunicodechar{Ⅸ}{IX}
\newunicodechar{Ⅹ}{X}
\newunicodechar{Ⅺ}{XI}
\newunicodechar{Ⅻ}{XII}
\newunicodechar{·}{$\cdot$}
%    \end{macrocode}
%
% \subsection{SI 单位}
%
%    \begin{macrocode}
\RequirePackage{siunitx}
\sisetup{
  inter-unit-product = \cdot,
  list-separator = {、},
  list-final-separator = {和},
  list-pair-separator = {和},
  list-units = single,
  range-units = single,
  range-phrase = \blx,
  separate-uncertainty = true
}
%    \end{macrocode}
%
% \subsection{页眉页脚设置}
%
%    \begin{macrocode}
\RequirePackage{fancyhdr}
\pagestyle{fancy}
\fancyhf{}
\fancyfoot[C]{\zihao{-5}\thepage}
\renewcommand{\headrulewidth}{0pt}
\renewcommand{\footrulewidth}{0pt}
\fancypagestyle{firstpage}{
    \fancyhf{}
    \renewcommand{\headrulewidth}{0pt}
    \renewcommand{\footrulewidth}{0pt}
}
%    \end{macrocode}
%
% \subsection{封面命令}
%
%    \begin{macrocode}
\RequirePackage{eso-pic}
\RequirePackage{ifthen}
\RequirePackage{calc}
\RequirePackage{tikz}
\newcommand{\@proposaltitle}{}
\newcommand{\@school}{}
\newcommand{\@major}{}
\newcommand{\@studentid}{}
\newcommand{\@grade}{}
\newcommand{\@studentname}{}
\newcommand{\@supervisor}{}
\newcommand{\@submityear}{}
\newcommand{\@submitmonth}{}
\newcommand{\@submitday}{}
\newcommand{\proposaltitle}[1]{\renewcommand{\@proposaltitle}{#1}}
\newcommand{\school}[1]{\renewcommand{\@school}{#1}}
\newcommand{\major}[1]{\renewcommand{\@major}{#1}}
\newcommand{\studentid}[1]{\renewcommand{\@studentid}{#1}}
\newcommand{\grade}[1]{\renewcommand{\@grade}{#1}}
\newcommand{\studentname}[1]{\renewcommand{\@studentname}{#1}}
\newcommand{\supervisor}[1]{\renewcommand{\@supervisor}{#1}}
\newcommand{\submityear}[1]{\renewcommand{\@submityear}{#1}}
\newcommand{\submitmonth}[1]{\renewcommand{\@submitmonth}{#1}}
\newcommand{\submitday}[1]{\renewcommand{\@submitday}{#1}}
\newlength{\titlewidth}
\newboolean{forcecentertitle}
\setboolean{forcecentertitle}{false}
\newcommand{\forcecentertitleon}{\setboolean{forcecentertitle}{true}}
\newcommand{\forcecentertitleoff}{\setboolean{forcecentertitle}{false}}
\newcommand{\smarttitle}[1]{%
    \settowidth{\titlewidth}{\zihao{4}#1}%
    \ifthenelse{\boolean{forcecentertitle}}{%
        \centering
        #1
    }{%
        \ifthenelse{\lengthtest{\titlewidth > 18em}}{%
            \raggedright
            #1
        }{%
            \centering
            #1
        }
    }
}
\newlength{\titleextrafill}
\newboolean{forcetitlenofill}
\setboolean{forcetitlenofill}{false}
\newcommand{\forcetitlenofillon}{\setboolean{forcetitlenofill}{true}}
\newcommand{\forcetitlenofilloff}{\setboolean{forcetitlenofill}{false}}
\newlength{\titleminusfill}
\newlength{\titleplusfill}
%    \end{macrocode}
%
% \begin{macro}{\makecover}
% 生成封面的命令。
%    \begin{macrocode}
\newcommand{\makecover}{
    \thispagestyle{firstpage}
    \settowidth{\titlewidth}{\zihao{4}\@proposaltitle}
    \ifthenelse{\lengthtest{\titlewidth > 21.1em}}{%
        \setlength{\titleextrafill}{1.1cm}
    }{%
        \setlength{\titleextrafill}{0cm}
    }
    \ifthenelse{\boolean{forcetitlenofill}}{%
        \setlength{\titleminusfill}{-1.9cm}
        \setlength{\titleplusfill}{1cm}
    }{%
        \setlength{\titleminusfill}{0cm}
        \setlength{\titleplusfill}{-0.1cm}
    }
    \begin{center}
        \vspace*{2.19cm}
        {\fontsize{28pt}{28pt}\selectfont 东北大学}
        \vspace{0.535cm}
        {\fontsize{28pt}{28pt}\selectfont 毕业设计(论文)开题报告}
        \vspace{4.35cm}
        {\zihao{4}
            \hspace*{0cm}\begin{tabular}{@{}r@{\hspace*{-0em}}l@{}}
                \makebox[5em][s]{题\hfill 目:} &
                \begin{minipage}[t]{18em}
                    \linespread{1.85}\selectfont
                    \smarttitle{\@proposaltitle}
                \end{minipage} \\[\dimexpr0.1cm+\titleextrafill+\titleminusfill\relax]
                \makebox[5em][s]{学\hfill 院:} &
                \begin{minipage}[t]{18em}
                    \centering\@school
                    \hrule height 0.7pt
                \end{minipage} \\[0.1cm]
                \makebox[5em][s]{专\hfill 业:} &
                \begin{minipage}[t]{18em}
                    \centering\@major
                    \hrule height 0.7pt
                \end{minipage} \\[0.1cm]
                \makebox[5em][s]{学\hfill 号:} &
                \begin{minipage}[t]{5.2025cm}
                    \centering\@studentid
                    \vspace{0.03cm}
                    \hrule height 0.7pt
                \end{minipage}
                \hfill
                年级\begin{minipage}[t]{2.45cm}
                    \centering\@grade
                    \hrule height 0.7pt
                \end{minipage}\\[0.1cm]
                \makebox[5em][s]{姓\hfill 名:} &
                \begin{minipage}[t]{18em}
                    \centering\@studentname
                    \hrule height 0.7pt
                \end{minipage} \\[0.1cm]
                \makebox[5em][s]{指导教师:} &
                \begin{minipage}[t]{18em}
                    \centering\@supervisor
                    \hrule height 0.7pt
                \end{minipage} \\
            \end{tabular}
        }
        \vspace{\dimexpr3.45cm-\titleextrafill+\titleplusfill\relax}
        {\heiti\zihao{3} 东北大学教务处印制}
        \vspace{0.05cm}
        {\heiti\zihao{3}\makebox[2em][c]{\@submityear} 年\makebox[2em][c]{\@submitmonth}月\makebox[2em][c]{\@submitday}日}
    \end{center}
    \begin{tikzpicture}[remember picture,overlay]
        \draw[line width=0.675pt] ([xshift=7.225cm,yshift=-13.32cm]current page.north west)
        -- ([xshift=16.12cm,yshift=-13.32cm]current page.north west);
    \end{tikzpicture}
    \ifthenelse{\boolean{forcetitlenofill}}{%
    }{%
        \ifthenelse{\lengthtest{\titlewidth > 21.1em}}{%
        \begin{tikzpicture}[remember picture,overlay]
            \draw[line width=0.675pt] ([xshift=7.225cm,yshift=-14.415cm]current page.north west)
             -- ([xshift=16.12cm,yshift=-14.415cm]current page.north west);
        \end{tikzpicture}
    }}
    \setcounter{page}{0}
    \newpage
}
%    \end{macrocode}
% \end{macro}
%
% \subsection{标题样式}
%
%    \begin{macrocode}
\ctexset{
    section = {
        format = \heiti\Large,
        beforeskip = 6pt,
        afterskip = 6pt,
    },
    subsection = {
        format = \heiti\large,
        beforeskip = 6pt,
        afterskip = 6pt,
    },
    subsubsection = {
        format = \heiti\normalsize,
        beforeskip = 6pt,
        afterskip = 6pt,
    }
}
%    \end{macrocode}
%
% \subsection{模块命令}
%
%    \begin{macrocode}
\RequirePackage{longtable}
%    \end{macrocode}
%
% \begin{macro}{\mokuai}
% 创建内容模块的命令。
%    \begin{macrocode}
\newcommand{\mokuai}[3][-4]{%
\setlength{\LTpre}{0pt}%
\setlength{\LTpost}{0pt}%
\begin{longtable}{|@{\hspace{0.5em}}m{1.802cm}@{\hspace{0.5em}}|@{\hspace{0.4em}}m{13.825cm}@{\hspace{0.8em}}|}
\hline
\centering\kaishu\zihao{-4}\vspace{0.5em}\textbf{#2}\vspace{0.5em} &
\zihao{#1}\vspace{0.5em}#3\vspace{0.7em} \\
\endfirsthead
\hline
 &
\endhead
\hline
\endfoot
\hline
\endlastfoot
\end{longtable}
}
%    \end{macrocode}
% \end{macro}
%
% \begin{macro}{\jindu}
% 工作进度安排表。
%    \begin{macrocode}
\newcommand{\jindu}[2][5]{%
\setlength{\LTpre}{0pt}
\setlength{\LTpost}{0pt}
\begin{longtable}{|@{\hspace{0.5em}}m{1.802cm}@{\hspace{0.5em}}|@{}m{14.325cm}@{}|}
\hline
\centering\kaishu\zihao{-4}\textbf{%
\begin{tabular}[b]{@{}>{\centering\arraybackslash}p{0.8cm}@{}}
    \textbf{工作进度安排} \\
\end{tabular}} &
\zihao{#1}
\begin{tabular}{@{}|>{\centering\arraybackslash}m{0.8cm}|>{\arraybackslash}m{3.68cm}|>{\arraybackslash}m{3.5cm}|>{\arraybackslash}m{4.6cm}|@{}}
    \textbf{阶段} & \makecell[c]{\textbf{起止日期}} & \makecell[c]{\textbf{任务}} & \makecell[c]{\textbf{提交的阶段成果}} \\ \hline
    #2
\end{tabular} \\
\endfirsthead
\hline
 &
\endhead
\hline
\endfoot
\hline
\endlastfoot
\end{longtable}
}
%    \end{macrocode}
% \end{macro}
%
% \begin{macro}{\jindunoleft}
% 无左侧标题的工作进度表。
%    \begin{macrocode}
\newcommand{\jindunoleft}[2][5]{%
\setlength{\LTpre}{0pt}
\setlength{\LTpost}{0pt}
\begin{longtable}{|@{\hspace{0.5em}}m{1.802cm}@{\hspace{0.5em}}|@{}m{14.325cm}@{}|}
\hline
\centering\kaishu\zihao{-4} &
\zihao{#1}
\begin{tabular}{@{}|>{\centering\arraybackslash}m{0.8cm}|>{\arraybackslash}m{3.68cm}|>{\arraybackslash}m{3.5cm}|>{\arraybackslash}m{4.6cm}|@{}}
    \textbf{阶段} & \makecell[c]{\textbf{起止日期}} & \makecell[c]{\textbf{任务}} & \makecell[c]{\textbf{提交的阶段成果}} \\ \hline
    #2
\end{tabular} \\
\endfirsthead
\hline
 &
\endhead
\hline
\endfoot
\hline
\endlastfoot
\end{longtable}
}
%    \end{macrocode}
% \end{macro}
%
% \subsection{图片命令}
%
%    \begin{macrocode}
\RequirePackage{graphicx}
%    \end{macrocode}
%
% \begin{macro}{\tu}
% 插入图片。
%    \begin{macrocode}
\newcommand{\tu}[2]{%
\vspace{0.25em}
\begin{minipage}{13.825cm}
\centering
\includegraphics[width=#1]{#2}
\end{minipage}
\vspace{0.25em}
}
%    \end{macrocode}
% \end{macro}
%
% \subsection{表格命令}
%
%    \begin{macrocode}
\RequirePackage{booktabs}
\RequirePackage{tabularx}
%    \end{macrocode}
%
% \begin{macro}{\biao}
% 插入表格。
%    \begin{macrocode}
\newcommand{\biao}[2]{%
  \begin{minipage}{\linewidth}
    \vspace{0.25em}
    \centering
    \renewcommand{\arraystretch}{1}
    \begin{tabularx}{\linewidth}{*{#1}{>{\centering\arraybackslash}X}}
      \toprule
      #2
      \bottomrule
    \end{tabularx}
  \end{minipage}%
  \vspace{0.05em}
}
%    \end{macrocode}
% \end{macro}
%
% \subsection{题注命令}
%
%    \begin{macrocode}
\RequirePackage{xstring}
\newcounter{tizhutu}
\newcounter{tizhubiao}
%    \end{macrocode}
%
% \begin{macro}{\tizhu}
% 图表题注。
%    \begin{macrocode}
\newcommand{\tizhu}[3][5]{%
  \IfStrEq{#2}{tu}{%
    \refstepcounter{tizhutu}%
    {\centering\zihao{#1}图\thetizhutu\hspace{1em}#3\par}%
  }{%
    \IfStrEq{#2}{biao}{%
      \refstepcounter{tizhubiao}%
      {\centering\zihao{#1}表\thetizhubiao\hspace{1em}#3\par}%
    }{%
      \PackageError{tizhu}{Unknown type: #2}{Use 'tu' or 'biao'}%
    }%
  }%
}
%    \end{macrocode}
% \end{macro}
%
% \subsection{评审表格}
%
%    \begin{macrocode}
\RequirePackage{pdfpages}
%    \end{macrocode}
%
% \begin{macro}{\huiyijiyao}
% 插入会议纪要。
%    \begin{macrocode}
\newcommand{\huiyijiyao}{%
    \includepdf[
      pages=1,
      width=\paperwidth,
      height=\paperheight,
      pagecommand={\thispagestyle{fancy}}
    ]{huiyijiyao.pdf}
}
%    \end{macrocode}
% \end{macro}
%
% \subsection{参考文献}
%
%    \begin{macrocode}
\RequirePackage[
    backend = biber,
    style = gb7714-2025,
    sorting = none,
    gbnamefmt = lowercase,
    gbnoauthor = false,
    gbpub = true,
    gbpunctwidth = bylan,
    url = true,
    doi = true
]{biblatex}
\addbibresource{references.bib}
%    \end{macrocode}
%
% \begin{macro}{\printbibrange}
% 按范围打印文献。
%    \begin{macrocode}
\newcounter{bibrangecount}
\newcommand{\printbibrange}[3][heading=none]{%
  \setcounter{bibrangecount}{0}%
  \defbibcheck{bibrange}{%
    \stepcounter{bibrangecount}%
    \ifnumgreater{\value{bibrangecount}}{#3}%
      {\skipentry}%
      {\ifnumless{\value{bibrangecount}}{#2}%
        {\skipentry}%
        {}%
      }%
  }%
  \printbibliography[check=bibrange, #1]%
}
%    \end{macrocode}
% \end{macro}
%
% 设置参考文献样式。
%
%    \begin{macrocode}
\setlength{\bibitemsep}{0pt}
\setlength{\biblabelsep}{7.4mm}
\setcounter{gbrefcompress}{3}
\def\gbpunctcommacite{\addcomma}
\def\gbpunctprl{(}
\def\gbpunctprr{)}
\def\CharParenLeft{\iffieldequalstr{userd}{chinese}{(}{(}}
\def\CharParenRight{\iffieldequalstr{userd}{chinese}{)}{)}}
\def\gbpunctdot{\unskip\mbox{.}\allowbreak}
%    \end{macrocode}
%
% \subsection{其他设置}
%
%    \begin{macrocode}
\newcommand{\suojin}{\hspace*{2em}}
\newcommand{\tips}[1]{{\zihao{-5}\textcolor{red}{#1}}}
\RequirePackage[%
draft=false,
colorlinks=true,
CJKbookmarks=true,
linkcolor=black,
citecolor=black,
urlcolor=black,
hyperindex,
linktoc=all]{hyperref}
%    \end{macrocode}
%
%    \begin{macrocode}
\endinput
%</class>
%    \end{macrocode}
%
% \Finale
\endinput
